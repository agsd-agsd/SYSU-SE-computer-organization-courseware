\documentclass[12pt]{ctexart}
\usepackage[margin=2.5cm]{geometry}
\usepackage{listings}
\usepackage{xcolor}
\usepackage{fontspec}
\usepackage{amsmath}
\usepackage{mathtools}
\usepackage{longtable}
\usepackage[hidelinks]{hyperref}
\usepackage{breqn}
\usepackage{amsthm}
\usepackage{enumitem}
\usepackage{titlesec}
\usepackage{tcolorbox} % 题目框
\usepackage{fancyhdr}
\usepackage{hyperref}
\hypersetup{urlcolor=blue}
\usepackage{tabularray}
\usepackage{diagbox}
\usepackage{bm}
\usepackage{float}
\usepackage{tikz}
\usepackage{graphicx}  % 用于插入图片
\usetikzlibrary{trees}
\UseTblrLibrary{booktabs}%特殊的引入宏包方式
\ctexset{
  section = {
    name = {,、},  % 控制编号和标点
    number = \chinese{section},
    aftername = \hspace{0pt} % 设置编号和标题之间的间距为 0
  }
}
\usepackage{appendix}
\usepackage{pgfplots}
\usetikzlibrary{shapes, arrows, positioning, fit, shapes.geometric}
\pgfplotsset{compat=1.18}
\raggedbottom % 允许页面底部不强制对齐
\UseTblrLibrary{asmath}
\renewcommand{\equationautorefname}{公式} % 修正为\equationautorefname
\renewcommand{\theoremautorefname}{定理}
\newtheorem{theorem}{定理}[section] % 按节编号
\titlespacing*{\section}{0pt}{1.5ex plus 1ex minus .2ex}{1ex plus .2ex}

\lstset{
    basicstyle=\ttfamily\small,      % 等宽字体,缩小字体
    keywordstyle=\color{blue},       % 关键字颜色
    commentstyle=\color{green!50!black}, % 注释颜色
    stringstyle=\color{red},         % 字符串颜色
    numbers=left,                    % 行号在左侧
    numberstyle=\tiny,               % 行号字体大小
    stepnumber=1,                    % 每行显示行号
    numbersep=5pt,                   % 行号与代码间距
    showspaces=false,                % 不显示空格
    showstringspaces=false,          % 不显示字符串中的空格
    frame=single,                    % 添加边框
    breaklines=true,                 % 自动换行
    breakatwhitespace=true,          % 只在空白处换行
    captionpos=b,                    % 标题位置(底部)
    escapeinside={*@}{@*},           % 允许在代码中插入LaTeX
    language=,                       % 默认不指定语言,需手动设置
}

% 定义特定语言样式
\lstdefinestyle{pythonstyle}{
    language=Python,
    morekeywords={sm, OLS, add_constant, numpy}, % 添加特定关键字
}
\lstdefinestyle{cppstyle}{
    language=C++,
    morekeywords={vector, cout, iostream}, % 添加特定关键字
}
\lstdefinestyle{matlabstyle}{
    language=Matlab,
    morekeywords={fitlm, plot, figure}, % 添加特定关键字
}

% 页眉页脚
\pagestyle{fancy}
% 清空所有默认的页眉页脚
\fancyhf{}
% 只保留页码在下方居中
\fancyfoot[C]{\thepage}
% 如果你不想章节首页有页眉页脚,可以再加下面这一行
\renewcommand{\headrulewidth}{0pt} % 去掉页眉线
\renewcommand{\footrulewidth}{0pt} % 去掉页脚线

\begin{document}

\section*{一、单项选择题}

\begin{enumerate}
  \item C.4 \\[2pt]
  解析:需满足 $2^r \ge k + r + 1$,此处 $k=8$。检验 $r=4$ 时 $2^4=16 \ge 8+4+1=13$,成立。

  \item C.F4H \\[2pt]
  解析:BA$_{16}$ = 10111010$_2$。原码表示中符号位为 1(负),数值位 0111010。算术左移一位后为 1110100,对应 F4H。

  \item B.双符号位不同。\\
  双符号位不同表示溢出。

  \item D.汉字机内码是统一编码(错误)。\\
  不同系统使用不同汉字内码(GBK、BIG5、Unicode 等)。

  \item B.128。\\
  $2^7 = 128$ 种不同字符。

  \item D.运算结果无法表示。\\
  定点溢出是结果超出表示范围。

  \item D.最高位进位与次高位进位异或为 1。\\
  补码加法中,若两者异或为 1 则溢出。

  \item B.不变,补 1。\\
  算术右移时符号位保持不变,高位补 1(负数)。

  \item A.上溢。\\
  阶码过大称为上溢。

  \item A.ACC。\\
  原码一位乘法中,ACC 用于存放乘积高位。
\end{enumerate}

\section*{二、填空题}

\begin{enumerate}
  \item 共需进行 $n+1$ 次移位操作。

  \item 机器数为:$\boxed{0\,1101\,010010010}$。\\
  说明:$73_{10} = 1001001_2 = 1.001001 \times 2^6$,阶码移码表示为 $6+7=13=1101_2$。

  \item $9B_{16}$ 的十进制值为 $155$,真值为 $155-127=28$。

  \item 10110。\\
  原 1011 有奇数个 1,为奇校验附位应为 0。

  \item 1。\\
  11001 有 3 个 1,按偶校验接收结果为 1(出错)。

  \item $P_1 = 0$。\\
  位置 1,3,5,7 异或结果应为偶,得 $P_1=0$。

  \item 海明位号 = 5。\\
  错误码 101$_2=5$。

  \item 共需 5 次移位操作。\\
  被除数含 1 位整数与 4 位小数,共 5 次移位。

  \item 最大负数真值:
  \[
    -\left(2 - 2^{-n}\right) \times 2^{2^{m}-1}
  \]

  \item 11101010。\\
  11110101 左移一位(低位补 0)得到 11101010。
\end{enumerate}

\section*{三、问答题}

\subsection*{(1) $x=-0.1011,\;y=-0.1101$,补码加法与溢出判断}

采用放大法,设定小数部分 4 位,放大 $2^4$:

\[
x=-11,\quad y=-13.
\]
5 位补码表示:
\[
-11 = 10101,\quad -13 = 10011.
\]
相加:
\[
10101 + 10011 = 1\,01000.
\]
丢弃最高进位,结果 $01000=8$。

还原为小数:$8/16=0.5=0.1000_2$。  
原为负数相加结果变正,发生溢出。

\[
\boxed{[x+y]_{\text{补}} = +0.1000,\text{发生溢出。}}
\]

\subsection*{(2) $x=0.1011,\;y=0.1101$,变形补码求 $x-y$}

放大 $2^4$:
\[
x=11,\quad y=13,\quad x-y=-2.
\]
5 位补码:$-2 = 11110$。  
还原:$-2/16=-0.0010_2$。

\[
\boxed{[x-y]_{\text{补'}}=-0.0010,\ \text{无溢出。}}
\]

\subsection*{(3) 原码一位乘法($x=-0.1110,\ y=-0.1101$)}

两数同号,结果为正。取幅值:
\[
0.1110\times 0.1101.
\]

\begin{tblr}{
  colspec={ccc},
  hlines,vlines,
  cells={c}
}
步骤 & 乘数位 & 部分积及说明 \\
1 & 1 (1/2) & $0.01110000 = A\times \tfrac{1}{2}$ \\
2 & 1 (1/4) & $0.00111000 = A\times \tfrac{1}{4}$ \\
3 & 0 (1/8) & $0.00000000$ \\
4 & 1 (1/16) & $0.00001110 = A\times \tfrac{1}{16}$ \\
\hline
合计 &  & $0.10110110 = 0.7109375$ \\
\end{tblr}

四位取值约为 $0.1011$,符号为正:
\[
\boxed{[x\cdot y]_{\text{原}} = +0.1011.}
\]

\subsection*{(4) 原码恢复余数法除法 ($x=-0.1001,\ y=0.1011$)}

取幅值:
\[
|x| = 0.1001 = 0.5625,\quad |y| = 0.1011 = 0.6875.
\]
真实商约为 $0.818181$。用恢复余数法取 4 位商。

\begin{tblr}{
  colspec={cccc},
  hlines,vlines,
  cells={c}
}
步骤 & 当前余数 & 比较结果 & 商位 \\
1 & 0.1001 & 小于除数 & 0 \\
2 & 1.0010 & $\ge$ 除数 & 1 \\
3 & 0.0111 & $\ge$ 除数后余 0.0111 & 1 \\
4 & 0.0011 & 小于除数 & 0 \\
\end{tblr}

综合得商 $\approx 0.1101$。  
由于被除数为负,最终原码:
\[
\boxed{[x/y]_{\text{原}} = -0.1101,\quad r \approx 0.0039\ (0.00000001_2).}
\]
验证:$(-0.8125)\times 0.6875 + 0.0039 \approx -0.5625$,正确。

\bigskip
\begin{center}
\fbox{全部题目解析完毕!}
\end{center}


\end{document}